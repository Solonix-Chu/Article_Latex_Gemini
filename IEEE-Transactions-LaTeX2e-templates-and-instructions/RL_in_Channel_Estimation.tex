\documentclass[journal]{IEEEtran}
\usepackage{amsmath,amsfonts}
\usepackage{algorithmic}
\usepackage{algorithm}
\usepackage{array}
\usepackage[caption=false,font=normalsize,labelfont=sf,textfont=sf]{subfig}
\usepackage{textcomp}
\usepackage{stfloats}
\usepackage{url}
\usepackage{verbatim}
\usepackage{graphicx}
\usepackage{cite}
\usepackage{booktabs}
\hyphenation{op-tical net-works semi-conduc-tor IEEE-Xplore}

\begin{document}

\title{应用强化学习于无线通信信道估计:\\当前研究、成就与创新前沿}

\author{IEEE Publication Technology,~\IEEEmembership{Staff,~IEEE}
\thanks{原始论文内容基于各种学术资源的整合与分析。}}

% 页眉设置
\markboth{无线通信中的强化学习应用, 2024}%
{应用强化学习于无线通信信道估计:当前研究、成就与创新前沿}

\maketitle

\begin{abstract}
本文综述了强化学习在无线通信信道估计中的应用。随着下一代无线网络场景的日益多样化,传统信道估计算法在复杂通信环境中面临挑战。强化学习作为一种数据驱动、模型自由的方法,为自适应信道估计提供了新的范式。文章首先介绍了信道估计中强化学习应用的背景,随后详细探讨了核心强化学习方法学,包括基于价值的方法和策略梯度方法。文章分析了强化学习在智能反射表面、时变与双色散信道、多天线系统以及端到端通信系统等多种通信场景中的应用。此外,本文总结了强化学习在信道估计中取得的显著成就和性能基准,指出了计算需求、可扩展性、数据效率和安全性等方面的挑战与局限。最后,文章展望了元强化学习、轻量级算法设计、鲁棒性增强以及跨层优化等创新前沿和研究方向,强调了强化学习在未来智能通信系统中的变革性影响。
\end{abstract}

\begin{IEEEkeywords}
强化学习, 信道估计, 无线通信, 智能反射表面, 深度强化学习, 端到端通信, 智能通信系统
\end{IEEEkeywords}

\IEEEpeerreviewmaketitle

\section{引言:强化学习在信道估计中的兴起}

\IEEEPARstart{在}{无}线通信系统中,信道估计扮演着至关重要的角色,其精度直接影响数据传输速率、比特误码率和信号覆盖等关键性能指标\cite{ref1}。随着下一代无线网络(NextG、5G、6G)的发展,应用场景变得日益多样化,包括大规模多输入多输出(Massive MIMO)、毫米波通信(mmWave)、智能反射表面(IRS)以及车联网(V2X)通信等,这些都对信道估计的精度和适应性提出了更严格的要求\cite{ref3}。

\subsection{无线系统中信道估计的演进格局}

传统信道估计算法,如最小二乘法(LS)和线性最小均方误差(LMMSE),往往难以准确捕捉复杂通信环境中的信道特性,特别是在存在显著多径效应、快速变化的信道状态或强干扰的环境中,导致估计精度下降\cite{ref1}。这种技术转变不仅仅是对现有算法的渐进式改进,更是对传统方法在日益复杂和动态的下一代网络环境中固有局限性的回应。传统方法通常依赖于高斯或瑞利等统计信道模型,这些模型在高度动态、非平稳或干扰严重的实际场景中可能不再适用\cite{ref1}。

虽然监督式深度学习可以学习这些复杂映射,但它需要大量针对每种可能信道条件的标记数据,这在实际中几乎不可行。

\subsection{强化学习:自适应信道估计的范式转变}

强化学习(RL)为自适应信道估计提供了一种新的范式。它是一种数据驱动、模型自由(或基于模型)的方法,智能体通过与环境的交互学习做出最优决策\cite{ref7}。这一特性使其特别适合于动态变化、未知的信道环境,在这些环境中,传统数学模型可能不准确或难以获取\cite{ref7}。强化学习能够在动态信道环境中持续优化决策,从而提高信道估计的鲁棒性\cite{ref1}。

强化学习在信道估计中的应用标志着通信系统向更高自主性和智能化的转变,系统能够自优化物理层功能。通过强化学习做出信道估计决策的智能体,例如选择导频符号\cite{ref12}、配置智能反射表面(IRS)\cite{ref1}或联合优化收发器\cite{ref10},代表了比预编程算法或离线训练的监督模型更高程度的自动化,这与AI原生6G网络的更广泛愿景相一致\cite{ref4}。

\subsection{强化学习相对于传统和常规机器学习方法的优势}

与传统信道估计算法和常规机器学习方法(如监督学习)相比,强化学习展现出多方面优势:

\begin{itemize}
\item \textbf{适应性:}强化学习的核心优势在于通过持续学习和策略调整,适应动态变化的环境的能力\cite{ref1}。与需要大量标记数据集(对所有信道条件通常难以获取)的监督学习不同,强化学习从经验和奖励中学习。
\item \textbf{无模型能力:}许多强化学习算法不需要信道的精确数学模型,这在信道模型复杂或未知的情况下尤为重要,特别是在端到端(E2E)系统或存在非线性特性时\cite{ref7}。虽然这种"无模型"特性为未建模动态提供了鲁棒性,但与基于模型的方法或具有精确标签的监督学习相比,它也可能导致更高的样本复杂性和确保收敛到全局最优的挑战。奖励函数设计变得至关重要,确保它引导智能体朝着所需的信道估计目标前进并非易事\cite{ref7}。
\item \textbf{长期性能优化:}强化学习旨在最大化累积奖励,从而制定长期最优的策略,而不仅仅是为了即时收益做出贪婪选择\cite{ref15}。
\item \textbf{处理非线性:}深度强化学习(DRL)结合了强化学习和深度神经网络,能够通过自动特征提取有效处理复杂信道环境中的非线性和动态波动\cite{ref1}。
\end{itemize}

\section{信道估计的核心强化学习方法学}

强化学习为信道估计提供了多样化的算法工具箱。这些方法可以广泛分为基于价值的方法和策略梯度方法,后者通常结合Actor-Critic架构。近年来,深度强化学习(DRL)通过引入深度神经网络作为函数逼近器,显著扩展了这些方法处理高维和复杂问题的能力,这对于现代通信系统中复杂的信道状态表示和连续控制参数空间至关重要\cite{ref8}。

\subsection{基于价值的方法:基础和应用}

基于价值的方法旨在学习一个价值函数,用以估计在特定状态下采取某一行动或处于某一状态的"价值"。

\subsubsection{Q-Learning和深度Q网络(DQN):原理和信道估计策略}

\paragraph{Q-Learning}
Q-Learning是一种无模型、异策略的强化学习算法,学习状态-动作价值函数(Q值),表示在给定状态下采取特定动作的预期累积奖励\cite{ref7}。它通常使用Q表来存储这些值。在信道估计中,Q-learning已在各种场景中应用。例如,在多输入单输出非正交多址接入(MISO-NOMA)系统中,基站的Q-learning智能体可以预测下行链路信道系数,以最大化系统总速率并减少估计损失\cite{ref15}。在这类应用中,状态可以包括当前用户信道条件,动作是对信道系数的预测或调整,奖励基于系统总速率\cite{ref15}。另一个应用是MIMO OFDM系统中的连续去噪,其中Q-learning通过识别信道曲率来识别不可靠的信道估计,并应用基于几何的去噪动作来更新信道估计。其状态是量化的信道估计,动作涉及选择用于去噪的子载波,奖励反映了均方误差(MSE)的减少\cite{ref17}。

\paragraph{深度Q网络(DQN)}
DQN是Q-learning的扩展,使用深度神经网络来近似Q值函数,使其能够处理大规模连续状态空间\cite{ref7}。DQN通常采用经验回放和目标网络等技术来增强学习稳定性。在NOMA系统的信道系数估计中,DQN用于预测信道参数,目标是最大化下行链路总速率\cite{ref18}。DQN克服了传统Q-learning在Q表存储方面的限制\cite{ref18}。此外,改进的算法如双重DQN(DDQN)可以减轻DQN中的过估计偏差\cite{ref19}。

\paragraph{性能洞察和固有限制}
Q-learning及其基于深度学习的变体在特定信道估计任务中取得了显著的性能改进。例如,它们在比特误码率(BER)、中断概率和系统总速率方面优于传统的MMSE方法\cite{ref15}。DQN能够有效处理高维状态空间\cite{ref19}。

然而,这些方法也有固有的局限性。当处理复杂问题时,Q-learning可能会因Q表过大而导致存储和计算困难\cite{ref15}。虽然DQN解决了状态空间问题,但它可能存在过估计偏差\cite{ref7},并且对超参数选择敏感\cite{ref19}。此外,这两种方法通常适用于离散动作空间,这对于优化信道估计中的许多连续参数(如精确相移)是一个限制\cite{ref7}。计算复杂性和收敛时间也是令人担忧的问题\cite{ref15}。

\subsection{策略梯度和Actor-Critic方法:高级策略}

策略梯度方法直接学习将状态映射到动作的策略函数。Actor-Critic方法是策略梯度方法的一种流行实现,包括学习策略的"演员"网络和评估策略有多好的"评论家"网络。

\subsubsection{深度确定性策略梯度(DDPG)}
DDPG是一种异策略、无模型的Actor-Critic算法,为连续动作空间设计。演员网络学习一个确定性策略(将状态映射到特定动作),评论家网络学习一个Q值函数(评估演员选择的动作)\cite{ref1}。DDPG在信道估计中的应用包括:使用CNN-GRU模型提取的特征作为状态输入来优化IRS反射参数,DDPG智能体不断学习和调整IRS配置以改善信号质量和覆盖\cite{ref1}。在端到端通信系统中,DDPG用于在未知信道条件下联合训练发射机和接收机,无需事先的信道状态信息(CSI),使用接收机的损失作为奖励信号\cite{ref10}。

\subsubsection{近端策略优化(PPO)}
PPO是一种同策略Actor-Critic算法,以其学习稳定性和样本效率而闻名。它使用裁剪替代目标函数来限制策略更新的幅度,从而避免过大的策略变化\cite{ref7}。应用实例包括:通过考虑信噪比(SNR)和先前的RIS相移来优化RIS相移、无人机高度和通信调度\cite{ref7};优化IEEE 802.11ah物联网网络中的受限访问窗口(RAW)参数以提高吞吐量\cite{ref10};以及优化综合感知和通信(ISAC)系统中的信道采样模式\cite{ref28}。

\subsubsection{双延迟DDPG(TD3)}
TD3是DDPG的扩展,旨在解决DDPG的过估计偏差并提高其稳定性。它通过裁剪双Q学习、延迟策略更新和目标策略平滑等技术实现这些改进\cite{ref7}。TD3的应用包括:通过考虑信道对本地信息的响应来优化RIS辅助网络中的RIS相移矩阵,以最大化总可实现有限块长率\cite{ref7};以及联合优化卫星波束成形、无人机定位和功率分配在集成卫星-无人机-RIS网络中\cite{ref30}。在\cite{ref21}中还提出了增强的TD3变体EE-DDPG。

\subsubsection{性能洞察和固有限制}
像DDPG和PPO这样的Actor-Critic方法非常适合连续动作空间,这对于信道估计中的许多参数(如波束成形向量、相移)至关重要\cite{ref1}。DDPG可以在无信道知识的端到端系统中实现显著性能\cite{ref10}。PPO提供稳定高效的学习过程\cite{ref16}。TD3改进了DDPG的稳定性和过估计问题\cite{ref7}。

然而,这些高级策略也面临挑战。DDPG可能存在过估计偏差,由于多个神经网络而具有高计算复杂性\cite{ref7}。作为同策略算法,PPO可能样本效率低。TD3在DDPG基础上增加了更多复杂性\cite{ref7}。此外,这些算法的超参数调整仍然是一个挑战\cite{ref19}。

RL算法的选择与信道估计问题的动作空间特性(离散与连续)密切相关。例如,在数据辅助信道估计中选择检测到的数据符号\cite{ref12}自然适合Q-learning/DQN的离散动作空间。相反,优化像IRS相移\cite{ref1}、波束成形向量或功率水平\cite{ref21}等连续参数,需要为连续动作空间设计的策略梯度或Actor-Critic方法,如DDPG、PPO或TD3\cite{ref7}。

\begin{table*}[!t]
\caption{信道估计中关键强化学习算法总结}
\label{tab:table1}
\centering
\begin{tabular}{|p{1.8cm}|p{1.5cm}|p{4.5cm}|p{4.5cm}|p{3cm}|}
\hline
\textbf{算法} & \textbf{类型} & \textbf{关键特性}(如离散/连续动作空间,同策略/异策略,无模型) & \textbf{信道估计中的MDP组件}(状态;动作;奖励) & \textbf{主要应用领域} \\
\hline
Q-Learning & 基于价值 & 离散动作空间,异策略,无模型 & \textbf{状态:}接收信号,导频数据,历史信道估计;\textbf{动作:}选择导频/数据符号,调整信道系数;\textbf{奖励:}MSE/NMSE减少,BER改善 & 数据辅助CE,信道预测 \\
\hline
DQN & 基于价值 & 离散动作空间,异策略,无模型,使用DNN逼近Q函数 & \textbf{状态:}高维信道特征,接收信号;\textbf{动作:}选择导频/数据符号,预测信道参数;\textbf{奖励:}总速率最大化 & NOMA系统信道估计 \\
\hline
DDPG & Actor-Critic & 连续动作空间,异策略,无模型 & \textbf{状态:}CSI特征,接收端损失信息;\textbf{动作:}调整IRS相位,设置波束成形向量;\textbf{奖励:}信号质量改善 & IRS优化,E2E学习 \\
\hline
PPO & Actor-Critic & 离散或连续动作空间,同策略,无模型,裁剪替代目标函数 & \textbf{状态:}SNR,先前RIS相移;\textbf{动作:}调整IRS相移/UAV高度/调度;\textbf{奖励:}信息时效性最小化 & RIS优化,IoT参数优化 \\
\hline
TD3 & Actor-Critic & 连续动作空间,异策略,无模型,解决DDPG过估计 & \textbf{状态:}本地CSI,系统状态;\textbf{动作:}优化RIS相移矩阵;\textbf{奖励:}有限块长率最大化 & RIS优化,资源管理 \\
\hline
\end{tabular}
\end{table*}

\section{实践中的强化学习:各种通信场景下的信道估计}

强化学习的适应性和学习能力使其能够解决各种通信场景中的独特信道估计挑战。从智能反射表面到高度动态的移动环境,以及新兴的毫米波和物联网应用,强化学习正被积极探索以提高信道估计的性能和鲁棒性。这些应用场景的特性,如信道动态性、硬件限制和系统目标,直接决定了马尔可夫决策过程(MDP)的制定——包括状态、动作和奖励的定义——以及相应RL算法的选择。

\subsection{智能反射表面(IRS)/可重构智能表面(RIS):RL驱动的优化}

\subsubsection{问题背景}
IRS/RIS技术利用大量低成本的无源反射单元智能地操纵无线传播环境,以增强接收信号质量或抑制干扰。然而,IRS/RIS的引入使信道估计变得复杂,通常涉及高维且难以直接估计和跟踪的级联用户-IRS-基站信道\cite{ref1}。传统方法难以适应IRS的动态配置需求\cite{ref2}。

\subsubsection{RL应用}
深度强化学习,特别是DDPG和PPO等算法,被广泛用于学习IRS/RIS反射单元的最优相移配置\cite{ref1}。RL智能体接收当前状态信息(如提取的信道特征、先前的相移配置、接收信号的SNR)并输出动作(即对相移的调整)。奖励函数通常基于接收信号质量、信干噪比(SINR)或系统总速率等指标设计。

\subsubsection{典型案例}
参考文献\cite{ref1}提出了一种智能信道估计模型,结合了卷积神经网络(CNN)和门控循环单元(GRU)提取的特征作为状态输入,并使用DDPG算法优化IRS配置。该模型在动态信道环境中不断学习和调整IRS配置策略,显著提高了信道估计的准确性和鲁棒性,超过了传统的LS和LMMSE方法。文献\cite{ref7}的综述也指出,DDPG可以基于CSI优化RIS相移,并研究了DQN、TD3和PPO在RIS中的应用\cite{ref7}。

\subsubsection{关键成就}
通过RL优化IRS/RIS可以显著增强SINR,提高信道容量,并在动态环境中实现自适应优化,从而提升整体通信系统性能\cite{ref1}。

\subsection{应对动态性:时变与双色散信道中的RL}

\subsubsection{问题背景}
无线信道本质上是时变的,特别是在高速移动场景中,信道特性变化迅速。双色散信道(其特性在时间和频率域都变化)更为复杂。这些快速变化对传统信道估计算法提出了重大挑战,因为它们通常假设信道在一段时间内是准静态的\cite{ref3}。

\subsubsection{RL应用}
RL用于在时变信道中选择检测到的数据符号作为额外的导频来辅助信道估计,从而更好地适应信道变化\cite{ref12}。通过将符号选择问题建模为MDP,RL算法可以计算最优符号选择策略。状态元素细化和滑动窗口等机制有助于跟踪信道的动态变化\cite{ref12}。

\subsubsection{典型案例}
参考文献\cite{ref12}提出了一种用于时变MIMO系统的RL辅助信道估计器,通过有效捕捉信道变化,表现优于传统估计器。该工作使用一阶高斯-马尔可夫过程建模信道,并根据信道变化调整数据符号的权重。参考文献\cite{ref14}也强调了类似方法用于时变MIMO系统,并指出由于计算复杂性和信道动态性,推导最优解非常困难。

\subsubsection{关键成就}
RL方法能够更有效地跟踪时间信道变化,在动态场景中表现优于传统估计算法\cite{ref12}。

\subsection{增强多天线系统:MIMO与大规模MIMO信道估计中的RL}

\subsection{RL在新兴前沿的探索:毫米波、物联网、V2X和水下信道估计}

\subsubsection{毫米波通信(mmWave)}
毫米波通信面临高路径损耗和对阻挡敏感等挑战。RL可用于联合优化波束成形和低分辨率ADC的ADC阈值在毫米波MIMO系统中\cite{ref34}。在RIS辅助毫米波系统中,统计CSI(S-CSI)可用于优化\cite{ref35}。

\subsubsection{物联网(IoT)/回传通信}
低功耗是物联网应用的关键要求。RL算法(如DQN、DDQN、DDPG、PPO)可以自适应调整传输参数(如调制方案、发射功率)并通过适应SNR和干扰变化来增强回传系统中的信道估计\cite{ref19}。例如,PPO已用于信道跳频或信道黑名单机制,以避免干扰并提高通信可靠性\cite{ref26}。

\subsubsection{车联网通信(V2X)}
V2X通信环境高度动态,信道快速变化。深度强化学习(包括单智能体RL(SARL)和多智能体RL(MARL))用于频谱共享和资源分配,这些决策隐含地依赖于对信道状态的理解\cite{ref9}。DRL还可以基于感知状态信息优化波束成形,从而减少对导频的依赖进行信道估计\cite{ref39}。

\subsubsection{水下通信(UWC)}
水下信道环境复杂,时变,并且经常遭受严重的多径效应。RL(如Q-learning、DQN)被探索用于基于(可能过时的)CSI的自适应调制\cite{ref41},以及优化传输参数\cite{ref7}。虽然在提供的文献中直接应用RL进行水下信道估计的研究不突出,但深度学习方法(如神经网络)已经应用于此\cite{ref43},RL可以补充这些方法。

在这些新兴领域,虽然RL有时隐式处理信道估计(如在E2E系统或资源分配任务中),但其潜力是巨大的。值得注意的是,与MIMO或IRS等领域相比,针对某些新兴领域(如V2X或UWC)的*直接*信道参数估计的RL框架可能不太成熟。这表明,在这些领域为信道估计任务本身开发更明确的RL方法可能是创新的有效途径。

\begin{table*}[!t]
\caption{不同通信场景下RL用于信道估计的应用}
\label{tab:table2}
\centering
\begin{tabular}{|p{2.2cm}|p{4cm}|p{3cm}|p{4cm}|p{3cm}|}
\hline
\textbf{通信场景} & \textbf{解决的信道估计问题} & \textbf{采用的RL方法} & \textbf{关键性能结果/改进} & \textbf{该场景中的挑战} \\
\hline
IRS/RIS & 级联信道估计,动态相移优化 & DDPG,PPO,DQN,TD3 & NMSE减少,SINR/总速率增加,动态适应 & 高维性,级联信道,快速变化 \\
\hline
时变/双色散信道 & 动态信道跟踪,数据符号辅助估计 & RL(一般),Q-Learning & 有效捕捉信道变化,优于传统估计器 & 快速时变,计算复杂性 \\
\hline
MIMO/大规模MIMO & 高维CSI获取,导频开销减少 & RL(低复杂度),Q-Learning & MSE/NMSE/BER减少,有限导频下性能提升 & 极高维度,导频污染 \\
\hline
端到端系统 & 隐式信道估计和无先验CSI的联合收发机优化 & DDPG & 无需显式信道模型的联合优化,复杂信道中BLER改善 & 黑盒信道,梯度传播 \\
\hline
\end{tabular}
\end{table*}

\section{RL信道估计的重要成就和性能基准}

将RL应用于信道估计已经产生了一系列令人鼓舞的结果,特别是在性能改进和环境适应性方面,展示了相对于传统方法和其他机器学习技术的明显优势。这些成就通常通过关键性能指标(KPI)进行量化,并与现有基准进行比较。

\subsection{相对于传统估计器(LS、LMMSE)和其他ML技术的量化改进}

研究表明,基于RL的框架,特别是DRL,在各种通信场景中实现的信道估计误差(如归一化均方误差-NMSE,均方误差-MSE)显著低于传统的LS和LMMSE方法\cite{ref1}。例如,使用DDPG的IRS辅助信道估计模型的信道估计误差远低于LS和LMMSE方法\cite{ref1}。在MISO-NOMA系统中,基于Q-learning的信道预测不仅改善了BER,而且实现了比MMSE甚至基于深度学习的LSTM方法更高的总速率\cite{ref15}。此外,用于MIMO OFDM系统的RL辅助去噪方法接近理想的LMMSE估计(需要完美的信道统计)并显著优于实际的LS估计\cite{ref17}。

这种性能优势,特别是在动态或复杂场景中(如IRS、时变信道),表明RL学习非线性映射和适应变化统计的能力是一个关键区别因素。LS/LMMSE通常依赖线性模型和二阶统计\cite{ref1},难以有效捕捉复杂信道现象(如多径、非线性失真、快速衰落)。DRL,以神经网络为函数逼近器,可以直接从数据中学习这些复杂关系\cite{ref1},导致文献\cite{ref1}等研究中展示的卓越性能。

\subsection{关键性能指标(KPI)和实现的收益}

\begin{itemize}
\item \textbf{比特误码率(BER)/块误码率(BLER):}RL方法已被证明可以改善BER\cite{ref13}。例如,DDPG-E2E系统在各种信道条件下实现了更好的BLER性能\cite{ref10}。
\item \textbf{均方误差(MSE)/归一化均方误差(NMSE):}RL信道估计中常报告的显著成就是MSE/NMSE的大幅减少\cite{ref1}。
\item \textbf{信干噪比(SINR)/总速率:}在RIS和NOMA等系统中,改善SINR和总速率往往是RL智能体的优化目标,并且已经实现了实际性能收益\cite{ref7}。
\end{itemize}

\subsection{在复杂和动态环境中验证的适应性和鲁棒性}

RL使模型能够在动态信道环境中持续优化决策,从而增强信道估计的鲁棒性\cite{ref1}。RL辅助估计器能够有效捕捉时变MIMO系统中的信道变化\cite{ref12}。DDPG-E2E系统在瑞西、瑞利和3GPP等复杂信道模型下展示了良好的鲁棒性\cite{ref10}。一个重要特性是RL方法可以在没有先验信道知识的情况下运行,适应未知的通信环境\cite{ref7}。

文献中频繁报告的"适应性"和"鲁棒性"成就是RL学习范式的直接体现——与环境交互并基于反馈(奖励)进行优化。与固定算法或在静态数据集上训练的监督模型不同,RL智能体基于当前(可能变化)环境中其行动的结果不断完善其策略\cite{ref1}。这种固有的反馈循环使其能够适应初始训练集中不存在的不可预见的信道条件或动态,从而导致观察到的鲁棒性\cite{ref1}。

RL在信道估计中的这些成就正为更加自主和自优化的无线网络铺平道路,减少了对每一种可能场景的手动干预和预定义模型的需求。RL在难以获取准确信道模型的场景(如E2E系统\cite{ref10},IRS\cite{ref1})或信道高度动态的场景\cite{ref12}中实现高性能的能力,指向能够自配置和自优化的系统。这与未来智能网络的目标一致,即能够管理复杂资源并适应不可预测条件,最小化人工监督。

然而,值得注意的是,虽然RL通常优于传统方法,但报告的收益通常特定于问题制定(MDP、奖励函数)和所使用的DRL架构。将这些收益推广到所有可能的信道估计任务或实现"即插即用"的RL解决方案仍然是一个开放的挑战。成功的实例(如\cite{ref1})涉及为特定问题(如IRS相位控制、符号选择、去噪)量身定制的精心设计的状态、动作和奖励。在IRS控制中表现出色的DDPG智能体\cite{ref1}可能无法在例如V2X信道跟踪中实现最佳性能,而无需对其学习框架进行重大重新设计。这突显出虽然RL*范式*本身很强大,但其*应用*仍然需要领域特定的专业知识。

\section{克服障碍:RL在信道估计应用中的挑战和局限性}

尽管RL在信道估计中展示了巨大潜力,但在其广泛部署到实际通信系统之前,需要解决几个关键挑战和固有限制。这些挑战涉及计算开销、可扩展性、数据效率、鲁棒性和安全性。

\subsection{计算需求:复杂性和延迟问题}

RL算法,特别是结合深度神经网络的深度强化学习(DRL),在训练和推理阶段都可能计算密集\cite{ref7}。计算最优策略可能涉及相当大的复杂性和延迟\cite{ref12}。例如,DDPG算法由于使用多个神经网络而有显著的计算开销\cite{ref7},而传统Q-learning的Q表在处理大状态空间时可能变得难以管理\cite{ref15}。这些计算需求对实际部署构成障碍,特别是在实时系统或资源受限的终端设备上\cite{ref11}。

\subsection{大规模部署中的可扩展性和实用性}

将RL解决方案扩展到具有大量用户、天线(如大规模MIMO)或IRS元素的系统仍然是一个艰巨的挑战\cite{ref7}。随着系统规模的增加,状态和动作空间可能呈指数级增长,使许多RL算法的学习过程变得难以处理\cite{ref13}。计算复杂性、可扩展性和数据效率的挑战往往相互交织。高度复杂的DRL模型可能提供更好的性能,但也会需要更多的数据和计算能力,阻碍其可扩展性。因此,研究人员必须在性能和实际可行性之间进行权衡,特别是在实际系统中。开发"轻量级"DRL\cite{ref46}直接针对这一系列挑战。

\subsection{MDP制定和奖励工程的关键作用}

定义适当的状态、动作,特别是奖励函数对于RL的成功至关重要,但并非易事\cite{ref7}。奖励函数必须准确反映信道估计的目标。不充分的MDP设计可能导致次优策略或收敛缓慢。DRL模型的"黑盒"性质,虽然对未知环境有益,但也增加了奖励工程和确保泛化的难度。缺乏清晰的DRL智能体决策过程分析模型使得设计完美的奖励函数更加困难\cite{ref7}。它还使得在新的、分布外场景中预测智能体行为更加困难,影响泛化能力\cite{ref11}。这与基于模型的优化形成对比,后者的行为更加可预测。

\subsection{数据效率、样本复杂性和泛化能力}

RL算法可能存在样本效率低的问题,意味着它们需要大量与环境的交互才能学习有效的策略\cite{ref11}。确保模型能够泛化到训练中未遇到的信道条件或与训练设置不同的环境至关重要\cite{ref3}。此外,DRL模型的离线训练通常需要大规模数据集\cite{ref46},在某些场景中可能难以获取。

\subsection{安全考虑:对对抗性操纵的脆弱性}

基于深度学习的信道估计模型,包括使用RL的模型,容易受到对抗性机器学习攻击\cite{ref4}。攻击者可能制造扰动,欺骗接收机得到错误的信道估计结果,构成严重的安全风险\cite{ref5}。随着ML/RL模型越来越成为通信系统的不可或缺部分,安全性成为一个新兴且关键的挑战。如果信道估计这一基本物理层过程可以通过对RL智能体的攻击被恶意操纵,整个通信链路可能受到危害。这超越了性能优化,进入了系统可信度和弹性的领域,这是下一代网络的关键主题\cite{ref4}。

克服这些挑战对于将基于RL的信道估计从学术研究转变为广泛实际部署至关重要。当前的局限性定义了该领域成熟所需的主要研究方向。这些领域的解决方案(如高效算法、用于数据效率的迁移/元学习、用于安全的鲁棒RL)将使RL成为通信工具包中的标准工具。

\section{前进之路:RL在信道估计中的创新前沿和研究方向}

尽管已取得显著进展,但RL在信道估计中仍有大量创新空间和需要探索的研究方向。未来研究趋势主要集中在提高实用性、适应性、效率、可信度,以及RL与其他通信模块的集成,以满足下一代无线通信系统日益增长的复杂性和多样化需求。

\subsection{增强适应性和效率的高级学习范式}

\subsubsection{元强化学习(Meta-RL):使能快速适应新信道环境}

\paragraph{概念} 
元RL,或"学习如何学习",旨在训练能够快速适应新任务的智能体,如不同的信道环境、用户移动性模式或新的服务质量(QoS)要求\cite{ref48}。这对于处理非平稳信道至关重要。

\paragraph{应用} 
模型可以被训练以找到最优初始参数,使其能够快速收敛到新的信道条件\cite{ref48}。模型无关元学习(MAML)是一种突出的算法\cite{ref48}。例如,文献\cite{ref48}将MAML与DQN结合,优化无人机通信中的飞行轨迹和调度策略,以最小化信息时效性(AoI)和传输功率的组合。参考文献\cite{ref49}引入了GMA,一种基于上下文的元RL方法,结合专家混合(MoE)来改善MAC协议快速适应的表示学习。

\paragraph{潜力} 
减少新环境中的训练开销,提高泛化能力\cite{ref48}。元RL还可以为快速适应调整RL解决方案的超参数\cite{ref48}。

\paragraph{相关研究} 
\cite{ref48}(MAML与DQN用于无人机AoI/功率优化),\cite{ref49}(GMA,用于MAC协议适应的基于上下文的元RL与MoE),\cite{ref54}(用于下行链路信道预测的元学习,使用少量数据适应新环境),\cite{ref52}(MAML用于时变OFDM信道估计,少样本快速适应),\cite{ref50}(离线元RL)。

\subsubsection{迁移学习(TL):跨通信任务和场景的知识迁移}

\paragraph{概念} 
利用从源任务/环境获得的知识来改善目标任务/环境中的学习,从而减少对目标域中大量数据的依赖\cite{ref8}。

\paragraph{应用} 
模型可以在各种环境的数据上预训练,然后使用来自新环境的有限数据进行微调\cite{ref54}。例如,文献\cite{ref54}在FDD大规模MIMO系统中提出了一种直接迁移算法,其中模型在来自所有先前环境的数据上训练,然后使用新环境中少量标记数据进行微调,以根据上行链路CSI预测下行链路CSI。迁移学习还可用于初始化异构联邦学习(HFL)中的本地模型参数\cite{ref55}。

\paragraph{潜力} 
加速学习过程,改进数据稀缺场景中的性能,更有效地适应新系统配置或信道统计。

\paragraph{相关研究} 
\cite{ref54}(FDD大规模MIMO信道预测的直接迁移和元学习),\cite{ref55}(TL用于增强RIS辅助小区无MIMO系统中信道估计的HFL初始化)。

\subsubsection{联邦和分布式强化学习:走向隐私保护和可扩展的信道估计}

\paragraph{概念} 
在多个设备(用户、基站)之间协作训练RL模型,无需共享原始本地数据,从而增强隐私\cite{ref4}。分布式RL允许智能体基于本地信息学习策略,同时贡献于全局目标。

\paragraph{应用} 
联邦学习可用于RIS辅助小区无MIMO系统中的信道估计,用户形成联盟并使用分布式DRL(如基于Qmix的方法)进行决策\cite{ref55}。

\paragraph{潜力} 
提高可扩展性,减少集中式训练的通信开销,保护用户数据隐私,支持设备上学习。

\paragraph{相关研究} 
\cite{ref55}(HFL与联盟形成和DRL用于信道估计),\cite{ref7}(提及FL/SL作为RIS的未来研究方向),\cite{ref4}(用于5G后系统的分布式学习基础信道估计模型)。

对高度动态和多样化6G场景(例如,不同服务、用户密度、信道条件)的快速适应需求是元RL和迁移学习研究的强大驱动力。为特定场景训练的传统RL,甚至DRL,在条件显著变化时可能性能严重下降。从头开始重新训练通常太耗时或数据密集\cite{ref49}。元RL\cite{ref48}和TL\cite{ref54}提供了"学习适应"或"迁移知识"的机制,这对于未来网络中设想的灵活性至关重要。

\subsection{设计轻量级和计算高效的DRL算法}

\paragraph{问题} 
DRL的高计算成本阻碍了其在资源受限设备上的部署\cite{ref11}。

\paragraph{方法}
\begin{itemize}
\item 开发更简单的神经网络架构,例如轻量级CNN与Transformer结合\cite{ref46}。
\item 探索无限宽度神经网络/卷积神经切线核(CNTK)等方法,可能不需要大型数据集或广泛训练\cite{ref46}。这些方法利用无限宽度网络的训练动态可以由闭式方程描述的特性,仅使用导频位置的已知值来估计缺失的信道响应。
\item 执行算法级优化,减少Q-learning或Actor-Critic方法的复杂性,如文献\cite{ref13}中提出的子块分区策略。
\item 将修剪、量化或知识蒸馏等模型压缩技术应用于DRL模型。
\end{itemize}

\paragraph{潜力} 
实现设备上RL信道估计,减少延迟,使RL在实时应用中更加实用。

\paragraph{相关研究} 
\cite{ref46}(CNTK用于信道估计,减少数据和计算需求),\cite{ref13}(通过子块分区实现低复杂度RL),\cite{ref16}(PPO用于IoT RAW优化,暗示关注效率)。参考文献\cite{ref20}提到DQN可以处理大状态空间,但计算效率仍然是一个问题。

轻量级DRL\cite{ref46}和联邦/分布式RL\cite{ref4}的推动是互补的。轻量级模型更容易部署在边缘设备上,这是许多联邦学习场景的先决条件。联邦学习涉及在本地设备上训练模型\cite{ref55}。如果这些设备资源受限(常见于IoT、V2X),DRL模型必须计算成本低。因此,轻量级DRL的进步直接促进信道估计中更有效和广泛的联邦学习应用。

\begin{table*}[!t]
\caption{RL在信道估计中的创新前沿和未来研究方向}
\label{tab:table3}
\centering
\begin{tabular}{|p{2.5cm}|p{4.5cm}|p{4.5cm}|p{4.5cm}|}
\hline
\textbf{创新领域} & \textbf{探索的特定RL技术/概念} & \textbf{对信道估计的潜在影响} & \textbf{需要解决的关键挑战} \\
\hline
元强化学习 & MAML,Reptile,离线元RL,基于上下文的元RL与MoE & 快速适应未知信道,减少新环境的训练开销,改进泛化 & 元任务定义,少样本学习的稳定性,计算复杂性 \\
\hline
迁移学习 & 参数迁移,实例迁移,特征表示迁移,与DRL结合 & 减少对大型目标域数据的依赖,加速学习,数据稀缺场景中性能提升 & 负迁移风险,衡量源/目标任务之间的相关性,知识迁移的有效性 \\
\hline
联邦/分布式RL & 联邦平均与RL,多智能体RL(MARL),Qmix,异构FL & 隐私保护,可扩展性,减少通信开销,支持设备上学习 & 通信开销,异质性管理,收敛保证,激励机制设计 \\
\hline
轻量级DRL & 简化NN架构,无限宽度NN(CNTK),算法优化,模型压缩 & 设备上部署,减少延迟,实时应用可行性提高 & 在降低复杂性的同时保持高性能,模型压缩与RL的集成 \\
\hline
鲁棒RL & 动态环境RL算法,鲁棒性目标/约束,与干扰抑制结合,安全RL & 在非平稳、高干扰、对抗性设置中可靠通信 & 定义适当的威胁模型,鲁棒性与性能权衡,对抗样本的生成和防御 \\
\hline
可解释RL & RL的XAI技术,策略/价值函数可视化,固有可解释架构,学习世界模型 & 增强系统信任,便于调试与改进,洞察学习的信道特性 & 可解释性与性能权衡,解释复杂RL模型的难度,缺乏标准化评估指标 \\
\hline
联合CE与跨层优化 & 联合导频与估计器训练,联合波束成形/功率/RIS配置,E2E学习 & 系统级性能增益,更集成和智能的通信系统 & 联合优化问题的复杂性,多目标冲突与权衡,跨层信息交换 \\
\hline
\end{tabular}
\end{table*}

\subsection{为非平稳、高干扰和对抗性环境构建鲁棒RL智能体}

\paragraph{问题} 
实际无线信道通常是非平稳的,受不可预测干扰的影响,并可能面临对抗性攻击\cite{ref1}。

\paragraph{方法}
\begin{itemize}
\item 采用为动态环境设计的RL算法,如文献\cite{ref1}中用于复杂条件下IRS调整的DDPG,以及文献\cite{ref12}中用于时变信道的RL方法。
\item 开发具有明确鲁棒性目标或约束的RL智能体。
\item 将RL与干扰抑制技术结合,例如,文献\cite{ref58}中DRL优化的速率分裂多址接入(RSMA)来减轻干扰。
\item 研究安全RL方法,防御对信道估计器的对抗性攻击\cite{ref4}。这可能涉及对抗性训练或鲁棒奖励函数设计。
\end{itemize}

\paragraph{潜力} 
确保即使在具有挑战性和敌对的无线环境中也能可靠的通信性能。

\paragraph{相关研究} 
\cite{ref1}(DDPG用于动态IRS调整),\cite{ref58}(DRL用于RSMA预编码以减轻干扰),\cite{ref4}(安全问题和对鲁棒模型的需求)。

\section{结论:强化学习对信道估计的变革性影响}

强化学习的引入正在深刻改变无线通信中信道估计的理论和实践。通过赋予通信系统前所未有的自学习和适应能力,RL不仅克服了传统估计算法在日益复杂的无线环境中的许多局限性,还为未来智能通信系统的构建奠定了坚实基础。

回顾RL对信道估计的贡献,其核心价值在于有效解决动态、模型未知或模型复杂的信道场景。无论是通过Q-learning及其变体在MIMO和NOMA系统中优化符号选择和资源分配\cite{ref15},还是使用DDPG和PPO等高级Actor-Critic方法在IRS/RIS配置\cite{ref1}和端到端通信\cite{ref10}中实现智能决策,RL都展示了在提高估计精度、增强系统鲁棒性和优化整体通信性能方面的巨大潜力。它在时变信道跟踪\cite{ref12}、高效大规模MIMO估计\cite{ref13}以及毫米波、物联网和V2X\cite{ref9}等新兴领域的初步探索,都预示着一个更加数据驱动、更加智能的物理层的形成。

展望未来,元RL、迁移学习和联邦学习等高级学习范式的应用将进一步增强RL在信道估计中的适应性和效率,使其能够快速响应新环境、利用先验知识并保护用户隐私\cite{ref48}。轻量级DRL算法的发展将促进其在资源受限设备上的部署,而对鲁棒RL和可解释AI的研究旨在构建更值得信赖和可理解的智能体\cite{ref5}。至关重要的是,RL驱动的联合信道估计和跨层优化,如端到端学习和通信与控制的协同,预示着通信系统设计哲学的潜在转变,从传统的模块化、分离优化到更整体、集成的AI原生设计\cite{ref10}。

尽管在计算复杂性、样本效率、奖励设计和安全性等方面仍存在挑战,但持续的研究和创新不断突破这些瓶颈。RL的最终目标是帮助构建能够自主感知、理解、适应和优化其运行状态的无线网络。随着这些创新方向的深入,RL有望在6G和未来通信系统中发挥核心作用,赋能真正智能、自主和高度弹性的无线连接,从而彻底改变我们设计和运营通信网络的方式。

\bibliographystyle{IEEEtran}
\bibliography{references}

\end{document} 